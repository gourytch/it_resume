\documentclass[11pt,a4paper]{moderncv}
\moderncvtheme[blue]{casual}
\usepackage[utf8]{inputenc}
\usepackage[scale=0.8]{geometry}

\usepackage[russian]{babel}
% \usepackage{amssymb}
% \usepackage{amsmath}
% \usepackage{amstext}

\firstname{Лев}
\familyname{Бабченко}
\email{lvbabchenko@gmail.com}

\makeatletter
\renewcommand*{\bibliographyitemlabel}{\@biblabel{\arabic{enumiv}}}
\makeatother

\nopagenumbers{}

%----------------------------------------------------------------------------------
%            content
%----------------------------------------------------------------------------------
\begin{document}
\maketitle

\section{Навыки}
\cvcomputer{Языки программирования}{C/C++, Python, SQL, PgSQL, XHTML, CSS, XPath,
                       JavaScript, Perl, Bash, Ada, Pascal, Tcl, Lua, Java, POVRay, \LaTeX}
           {Операционные Системы}{Linux, Windows, HP-UX, AIX, \\
                               Solaris, xBSD, OS/2, BeOS, xDOS, CP/M, RT11}
\cvcomputer{СУБД}{DB2 UDB (SQL, API, Low-Level),\\
                       PostgreSQL, SQLite, MySQL,\\
                       OpenLDAP, dBase, Oracle, \\
                       BerkleyDB, Borland DBE}
           {Инструменты и библиотеки}{TSM, AWS, AJAX, Tcl/Tk, STL, wxWidgets, Django, WSGI,
                        SDL, OpenGL, X11, OpenLook,\\
                        Subversion, CVS, RCS, git, VSS, \\
                        RockeTrack, BugZilla, mantis, trac}
\cvcomputer{Ассемблеры}{BASM, MASM, TASM, GAS, Micro}{Instruction Sets}{x86, i8080A, PDP-11, p-code}

\section{Языки}
\cvlanguage{Русский}{Родной}{}
\cvlanguage{Английский}{Уверенный}{технический свободно, разговорный средний}

\section{Опыт работы}
% \subsection{Vocational}
\cventry{2009--наст.вр.}{Руководитель отдела ИТ}{ООО «ТиДжи Телеком»}{Челябинск}{}{Разработка ПО, решение проблем, управление отделом, учёт}
\cventry{2009--наст.вр.}{Системный администратор}{ООО «Прикладные Технологии»}{Челябинск}{}{Проект финансового анализа и прогнозирования}
\cventry{2004--2009}{Ведущий программист}{ООО «ТиДжи Телеком»}{Челябинск}{}{Разработка ПО для системы платёжных терминалов}
\cventry{2005--2006}{Программист}{oDesk Corp.}{San Francisco}{}{Web-Программирование}
\cventry{2003--2009}{Программист}{ООО «Прикладные Технологии»}{Челябинск}{}{Разработка системы анализа и восстановления данных СУБД DB2}
\cventry{1999--2003}{Ведущий программист}{Высшая Школа Бизнеса}{Челябинск}{}{Разработка инструмента для отображения и анализа данных для СУБД DB2}
\cventry{1996--1999}{Программист}{ЗАО «Недра»}{Челябинск}{}{Разработка Геоинформационной системы}
\cventry{1994--1996}{Программист}{ОВО при Курчатовском РОВД}{Челябинск}{}{Разработка ПО для системы радиоохраны}
\cventry{1993--1994}{Технический директор}{Рекламное агенство «Верктекс»}{Челябинск}{}{менеджмент ИТ-ресурсов, графика и оцифровка}
\cventry{1992--1992}{Программист}{ОНД Челябинской обл.}{Челябинск}{}{Разработка экспертных систем}
% \subsection{Miscellaneous}

\section{Проекты}

\cventry{2010--2010}{GDS}{}{}{}{Учёт транспортного и пассажиропотока -- модули управления и статистики}
\cventry{2010--2010}{DataPump}{}{}{}{Подсистема асинхронной загрузки данных}

\cventry{2010--2011}{TGPay2}{}{}{ООО «ТиДжи Телеком»}{Платёжная система, продолжение TGPay}
\cventry{2009--2011}{Tracker.KCN}{}{}{ООО «ТиДжи Телеком»}{Система ведения заказов}
\cventry{2009--2010}{BezzaBot}{}{}{}{Система полуавтоматической симуляции присутствия человека в сети}
\cventry{2007--current}{(unnamed)}{Applied Technologies, Ltd.}{}{}{Система автоматического сохранения данных}
\cventry{2005--2010}{TGPay}{}{}{ООО «ТиДжи Телеком»}{Система платёжных терминалов}
\cventry{2004--2005}{FunFactory}{}{}{ООО «ТачГейм»}{системное и прикладное ПО для игровых терминалов}

\cventry{2002--2009}{Recovery Expert for Multiplatforms}{}{}{Rocket Software, Inc.}{Инструментарий для аудита и восстаноаленния данных СУБД DB2}
\cventry{1999--2002}{QMF for Windows/Rocket Shuttle}{}{}{Rocket Software, Inc.}{Генератор отчётов и визуализатор данных для DB2}
\cventry{1999--1999}{TreeDB}{Высшая Школа Бизнеса}{Челябинск}{}{Древовидная база данных (LDAP-like)}

\cventry{2004--2005}{(unnamed)}{}{}{oDesk Corp.}{Платформа для проведения онлайн-аукционов}

\cventry{1997--1998}{NedraGIS}{}{}{ЗАО «Недра»}{Геоинформационная система}
\cventry{1997--1998}{GISImport}{}{}{ЗАО «Недра»}{преобразователь данных для NedraGIS}
\cventry{1997--1998}{GISPic}{}{}{ЗАО «Недра»}{полуавтоматический преобразователь крупноформатных изображений}

\cventry{1995--1997}{RCSS}{}{}{ОВО Курчатовского РОВД}{ПО для центрального пульта радиоохраны}
\cventry{1995--1997}{NetBackup}{}{}{ОВО Курчатовского РОВД}{Система сетевого резервного копирования}

\cventry{1993--1995}{ExpSys}{ОНД Челябинской обл.}{Челябинск}{}{
Экспертная Система нахождения возможных диагнозов на основе неполного множества анализов и обследований}

\cventry{1992--1993}{Scheduler}{ПТУ}{Челябинск}{}{
Приложение для полноэкранного составления школьных расписаний
    }

% \section{Education}
% \cventry{2002--2006}{Applied Mathematics}{Chelyabinsk State University}{}{\textit{unfinished}}{}
% \section{Master thesis}
% \cvline{title}{\emph{Title}}
% \cvline{supervisors}{Supervisors}
% \cvline{description}{\small Short thesis abstract}


\end{document}
%% end of file
